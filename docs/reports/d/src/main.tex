\documentclass[14pt, a4paper]{extreport}

% increase toc depth up to subsubsection
\setcounter{tocdepth}{3}
\setcounter{secnumdepth}{3}

\usepackage{tabu}
\usepackage{color}
\usepackage{array}
\usepackage{epstopdf}
\usepackage[russian]{babel}
\usepackage[utf8x, utf8]{inputenc}
\usepackage{tikz}
\usetikzlibrary{positioning, shapes.misc, shapes.geometric, arrows, shapes.multipart, shapes.arrows}
\usepackage{pgfplots}
\usepgfplotslibrary{dateplot}
\usepackage{amsmath}
\usepackage{algorithm2e}
\usepackage{algorithmic}
\usepackage{xcolor, colortbl}
\usepackage{mdframed}
\usepackage{textcase}
\usepackage{tabularx}
\usepackage[T2A,T1]{fontenc}
% 
\usepackage{indentfirst}
% setup  page fields
\usepackage{geometry}
\geometry{left=25mm}
\geometry{right=20mm}
\geometry{top=20mm}
\geometry{bottom=20mm}
% set line interval
\usepackage{setspace}
\onehalfspacing
% set footnotesize to 12 pt (for normalsize equal to 14 pt)
\renewcommand{\footnotesize}{\small}

\usepackage{titlesec}
\titleformat{\chapter}{\filcenter\bfseries}{\thechapter.}{1em}{}
\titleformat{\section}{\filcenter\bfseries}{\thesection}{1em}{}
\titleformat{\subsection}{\filcenter\bfseries}{\thesubsection}{1em}{}
\titleformat{\subsubsection}{\filcenter\bfseries}{\thesubsubsection}{1em}{}
\titleformat{\paragraph}{\filcenter\bfseries}{\paragraph}{1em}{}
\titlespacing*{\chapter}{0pt}{10pt}{10pt}

% make page number in top right corner
\usepackage{fancyhdr}
% rename abstract
\AtBeginDocument{\addto\captionsenglish{\def\abstractname{\MakeTextUppercase{Реферат}}}}
% counters
\usepackage[figure, table, page, enumiv]{totalcount}
%
\addto\captionsrussian
{
  \renewcommand{\contentsname}
    {\hfill{\normalsize\MakeTextUppercase{Содержание}}\hfill}
}
%
\addto\captionsrussian
{
  \renewcommand{\bibname}{\hfill\normalsize\MakeTextUppercase{Список использованных источников}\hfill}
}

\usepackage{tocloft}
% add dots for chapters
\renewcommand{\cftchapleader}{\cftdotfill{\cftdotsep}}
\renewcommand\cftchapfont{\mdseries}
\renewcommand\cftchappagefont{\mdseries}
% add dot after chapter number
\renewcommand{\cftchapaftersnum}{.}
%
\usepackage{caption}
\usepackage{subcaption}
\captionsetup[figure]{labelsep=space}
\captionsetup[table]{labelsep=space}

\usepackage{etoolbox}
\makeatletter
\patchcmd{\chapter}{\if@openright\cleardoublepage\else\clearpage\fi}{}{}{}
\makeatother

%
% flow chart commands
\tikzstyle{startstop} = [rectangle, rounded corners, minimum width=3cm, minimum height=0.5cm, text centered, draw=black, fill=red!30]
\tikzstyle{process} = [rectangle, minimum width=3cm, minimum height=0.5cm, text centered, draw=black, fill=orange!30]
\tikzstyle{decision} = [diamond, aspect=2, minimum width=1mm, minimum height=1mm, text centered, draw=black, fill=green!30]
\tikzstyle{arrow} = [thick,->,>=stealth]
%

% declare new operator for correct \limits usage
\DeclareMathOperator*{\argmin}{arg\,min}

\title{}
\date{}

\begin{document}
\RestyleAlgo{boxruled}
% style for top right corner page number
\fancypagestyle{plain}{%
    \fancyhf{} % clear all header and footer fields
    \fancyhead[R]{\thepage} % except the right top corner
    \renewcommand{\headrulewidth}{0pt} % remove line between header and main text
}
\pagestyle{plain}

\renewcommand\abstractname{\MakeTextUppercase{Реферат}}

\begin{titlepage}

\newcommand{\StudentName}{Данилычев Иван}
\newcommand{\Group}{1O-106М}
\newcommand{\CourseName}{Информационный поиск}
\newcommand{\LabNum}{1}
\newcommand{\Subject}{Поиск по постам сообществ ВКонтакте}
\newcommand{\PrepName}{Калинин А.\,Л.}

\newpage

\begin{center}
Московский Авиационный Институт \\*
(национальный исследовательский университет) \\*
Факультет прикладной математики и физики \\*
\hrulefill
\end{center}

\begin{center}
Кафедра вычислительной математики и программирования
\end{center}

\vspace{6em}

\begin{center}
\Large \CourseName \\
	Курсовой проект \\
  <<\Subject>>
\end{center}

\vspace{2em}
\vspace{6em}

\begin{flushright}
	\StudentName, \\
	группа: \Group \\
\vspace{1em}
руководитель:\\
   \PrepName \\
\end{flushright}

\vspace{\fill}

\begin{center}
Москва, 2016
\end{center}

\end{titlepage}


\newpage
\vspace*{-25mm}
\tableofcontents
\newpage

% Examples:
%\addcontentsline{toc}{section}{Заголовок}
%\section*{Заголовок}

%\addcontentsline{toc}{section}{Введение}
%\section{Введение}
\chapter{\MakeTextUppercase{Обкачка}}
Обкачал первые 5 миллионов групп (не все из них оказались открыты для роботов)
и все сообщения в них, без комментариев. 

Обкачивал с помощью скрипта на языке программирования Python параллельно
в 32 процесса с помощью консольной утилиты xargs с ключем -P.

\begin{verbatim}
    xargs -I{} -P32 bash -c "cat {} |\
     > python vk_crawler.py > {}.out 2> {}.log"
\end{verbatim}

\chapter{\MakeTextUppercase{Индексация}}
Железо: 1 машина, 8 vCPU 2.50GHz, 32GB, 300GB SSD.
Тестовые данные: $\sim$ 65GB, индексация производилась в 8 потоков.
Оценил время на обработку тестовой части данных для разных реализаций
на двух языках программирования:

\begin{itemize}
  \item Python: около 5 часов.
  \item С++: 35 минут.
\end{itemize}

Время индексации на языке программирования Python показалось неприемлимым, поэтому
дальнейшую разработку решил вести на языке программирования C++.

Итог: индексация по частям: $\sim$ 1.5 часа на все данные + merge 18 минут, размер индекса 75GB.

Сам индекс представляет собой отсортированный mmap-ленный файл по хэшам термов.
Для хеширования термов использовалась murmur2 хэш-функция~\cite{murmur2}.

В качестве записи использовалась упорядоченная четверка:
\begin{enumerate}
\item Хэшированный термин (8 байт).
\item Идентификатор группы (4 байта).
\item Идентификатор документа в группе (4 байта).
\item Позиция вхождения в документ (4 байта).
\end{enumerate}

Можно было бы использовать меньшие типы, например, для хранения позиции вхождения,
но это потребовало бы дополнительной проверки, что все вхождения действительно вмещаются в меньший тип.
Я решил, что выигрыш в размере будет небольшим и отказался от этой идеи.

\chapter{\MakeTextUppercase{Ранжирование и поиск}}
Списки пересекаются с разной величиной окна $f\times |Q|, f = 1\dots4$,
пока не наберется достаточно вхождений (1000).

В качестве функции ранжирования была выбрана BM25~\cite{wiki_bm25} со сглаженными
idf'ами посчитанными по индексу.

\newpage
\clearpage
%
\addcontentsline{toc}{chapter}{\MakeTextUppercase{Список использованных источников}}
\begin{thebibliography}{}
\bibitem{wiki_bm25} https://en.wikipedia.org/wiki/Okapi\_BM25 
\bibitem{boost_asio} http://www.boost.org/doc/libs/1\_55\_0/doc/html/boost\_asio\\
  /example/cpp11/echo/async\_tcp\_echo\_server.cpp 
\bibitem{murmur2} https://sites.google.com/site/murmurhash 
\end{thebibliography}

\end{document}
